\def\nodeseptwo{4cm}
\def\nodesize{.45cm}
\def\numhiddentwo{3}


%\newlength{\arrowsize}  
%\pgfarrowsdeclare{biggertip}{biggertip}{  
%  \setlength{\arrowsize}{1pt}  
%  \addtolength{\arrowsize}{.5\pgflinewidth}  
%  \pgfarrowsrightextend{0}  
%  \pgfarrowsleftextend{-5\arrowsize}  
%}{  
%  \setlength{\arrowsize}{0.4pt}  
%  \addtolength{\arrowsize}{.5\pgflinewidth}  
%  \pgfpathmoveto{\pgfpoint{-5\arrowsize}{4\arrowsize}}  
%  \pgfpathlineto{\pgfpointorigin}  
%  \pgfpathlineto{\pgfpoint{-5\arrowsize}{-4\arrowsize}}  
%  \pgfusepathqstroke  
%} 

\begin{tabular}{ccc}
Standard deep net architecture & & Input-connected architecture \\
\bardist
\begin{tikzpicture}[draw=black]

    \tikzstyle{neuron}=[circle,minimum size=17pt, draw = black, fill = white, thick]
    \tikzstyle{input neuron}=[neuron, fill=green!50];
    \tikzstyle{output neuron}=[neuron, fill=red!50];
    \tikzstyle{hidden neuron}=[neuron, fill=blue!50];
    \tikzstyle{pile} =[ultra thick, ->, >=stealth', shorten <= 0.6cm, shorten >= 0.6cm, -biggertip, line width = 2.5pt];

    % Define the input layer node
    \coordinate (I) at (0, 0);


    % Define the hidden layer nodes
    \foreach \name / \y in {1,...,\numhiddentwo}
    {
        \coordinate (H-\name) at (\nodeseptwo*\y, 0);
    }

    \path[pile] (I) edge (H-1) {};
    % Connect every node            
    \foreach \name in {2,...,\numhiddentwo}
    {
	 %\path[pile] (I) edge (H-\name) {};
	 \pgfmathsetmacro\hindex{\name - 1}
	 \path[pile] (H-\hindex) edge (H-\name) {};
         %\path[pile] (I) edge [bend left] (H-\name) {};
    }

    \draw (I) node[neuron, ultra thick] {};
    \draw (I) node[below = 0.5cm]  {$\vx$};

    % Draw the hidden layer nodes
    \foreach \name / \y in {1,...,\numhiddentwo}
    {
	\draw (H-\name) node[neuron, ultra thick]  {};
        \draw (H-\name) node[below = 0.34cm] {$\vf^{(\y)}$};
    }
\end{tikzpicture} & \hspace{2cm} &
\bardist
\begin{tikzpicture}[draw=black]
    \tikzstyle{neuron}=[circle,minimum size=17pt, draw = black, fill = white, thick]
    \tikzstyle{input neuron}=[neuron, fill=green!50];
    \tikzstyle{output neuron}=[neuron, fill=red!50];
    \tikzstyle{hidden neuron}=[neuron, fill=blue!50];
    \tikzstyle{pile} =[ultra thick, ->, >=stealth', shorten <= 0.6cm, shorten >= 0.6cm, -biggertip, line width = 2.5pt];

    % Define the input layer node
    \coordinate (I) at (0, 0);


    % Define the hidden layer nodes
    \foreach \name / \y in {1,...,\numhiddentwo}
    {
        \coordinate (H-\name) at (\nodeseptwo*\y, 0);
    }

    \path[pile] (I) edge (H-1) {};
    % Connect every node            
    \foreach \name in {2,...,\numhiddentwo}
    {
	 \pgfmathsetmacro\hindex{\name - 1}
	 \path[pile] (H-\hindex) edge (H-\name) {};
         \path[pile] (I) edge [bend left] (H-\name) {};
    }

    \draw (I) node[neuron, ultra thick] {};
    \draw (I) node[below = 0.5cm]  {$\vx$};

    % Draw the hidden layer nodes
    \foreach \name / \y in {1,...,\numhiddentwo}
    {
	\draw (H-\name) node[neuron, ultra thick]  {};
        \draw (H-\name) node[below = 0.34cm] {$\vf^{(\y)}$};
    }
\end{tikzpicture}
\end{tabular}
